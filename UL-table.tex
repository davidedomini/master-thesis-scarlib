\begin{table}
    \centering
    \begin{tabularx}{\textwidth}{lX}
        \hline
        \textbf{\emph{Concept}} & \textbf{\emph{Definition}} \\
        \hline
        Environment & The context in which the \emph{agents} are immersed and operate,
                        it is a representation of the task to be solved.
                        It is capable of interacting with the agents, providing information
                        about its current \emph{state}, receiving the \emph{actions} that one or more agents
                        wish to execute, and returning the corresponding \emph{reward}. \\
        \hline
        Agent       & An entity that interacts within an \emph{environment} and with other \emph{agents}
                        in order to learn the optimal \emph{actions} sequence to maximize a \emph{reward} signal. 
                        It is equipped with \emph{sensors}, \emph{actuators} and a \emph{communication mechanism}. \\
        \hline
        State       & A representation of the \emph{environment} at a given time, 
                        including any relevant information that an \emph{agent} can perceive 
                        or use to make decisions about its \emph{actions}. \\
        \hline
        Action      & A decision or choice made by an \emph{agent} in response to the current 
                        state of the \emph{environment}. \\
        \hline
        System      & A collection of \emph{agents} that interact within a shared \emph{environment}. 
                        It defines the training and execution flow of the agents. \\
        \hline
        Policy      & A function that maps the current \emph{state} of the \emph{environment} to a probability 
                        distribution over the set of possible \emph{actions} that the \emph{agent} 
                        can take in that \emph{state}. The policy specifies the \emph{agent's} behavior 
                        or strategy in response to different \emph{states} of the \emph{environment}, 
                        and it is learned through a process of trial and error using the 
                        \emph{reward} signal as feedback. \\
        \hline
        Sensor     & A mechanism that allows an \emph{agent} to perceive certain characteristic of 
                            the \emph{environment}. \\
        \hline
        Actuator   & A mechanism that allows an \emph{agent} to perform certain actions on the 
                            \emph{environment}. \\
        \hline
    \end{tabularx}
    \caption{Hybrid AC-MARL approach ubiquitous language}
    \label{tab:ul}
\end{table}