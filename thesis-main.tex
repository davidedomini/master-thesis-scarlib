%%%%
% Consiglio la visione dei seguenti tutorial:
% - https://www.youtube.com/watch?v=ihxSUsJB_14
% - https://www.youtube.com/watch?v=XTFWaV55uDo
%%%%
\documentclass[12pt,a4paper,openright,twoside]{book}
\usepackage[utf8]{inputenc}

%\newcommand{\thesislang}{italian} % decommentare in caso di tesi in italiano
\newcommand{\thesislang}{english} % commentare in caso di tesi in italiano
\usepackage{thesis-style}
% version
\newcommand{\versionmajor}{0}
\newcommand{\versionminor}{1}
\newcommand{\versionpatch}{2}
\newcommand{\version}{\versionmajor.\versionminor.\versionpatch}
\typeout{Document version: \version}

\begin{document}
	
\frontmatter

% ! TeX root = thesis-main.tex
\title{Title}
\author{Candidate Name Here}
\date{\today}

\newgeometry{margin=0.8in}
\begin{titlepage}
	\begin{center}
		% \vspace*{0.2cm}
		
		\large
		\textbf{ALMA MATER STUDIORUM -- UNIVERSITÀ DI BOLOGNA \\ CAMPUS DI CESENA}
		\\
		\noindent\hrulefill
		\vspace{0.4cm}
		
		\Large
		Scuola di Ingegneria e Architettura \\
		Corso di Laurea Magistrale in Ingegneria e Scienze Informatiche
		
		\Huge
		\vspace{4cm}
		\textbf{
			Aggregate Computing and Many-Agent Reinforcement Learning: 
			Towards a Hybrid Approach
		}
		
		\large
		\vspace{1cm}
		Tesi di laurea in 
		\\ 
		\textsc{Pervasive Computing}
		
		\vspace{5.5cm}
		\begin{minipage}[t]{0.64\textwidth}
			\begin{flushleft}
				\textit{Relatore} 
				\\ 
				\textbf{Prof.} \textbf{Mirko Viroli}
				\\
				\vspace{0.4cm}
				\textit{Correlatore} 
				\\
				\textbf{Dott.} \textbf{Gianluca Aguzzi}
			\end{flushleft}
		\end{minipage}
		\begin{minipage}[t]{0.34\textwidth}
			\begin{flushright}
				\textit{Candidato} 
				\\ 
				\textbf{Davide Domini}
			\end{flushright}
		\end{minipage}\\
		
		\vfill
		\noindent\hrulefill
		\vspace{0.3cm}
		\Large
		
		Seconda Sessione di Laurea
		\\
		Anno Accademico 2022-2023
	\end{center}
\end{titlepage}
\restoregeometry


\begin{abstract}	
Max 2000 characters, strict.
\end{abstract}

\begin{dedication} 
Optional. Max a few lines.
\end{dedication}

\begin{acknowledgements}
Optional. Max 1 page.
\end{acknowledgements}

%----------------------------------------------------------------------------------------
\tableofcontents   
\listoffigures     
\lstlistoflistings 
%----------------------------------------------------------------------------------------

\mainmatter

%----------------------------------------------------------------------------------------
\chapter{\introductionname}
\label{chap:introduction}
%----------------------------------------------------------------------------------------


\paragraph{Thesis motivation} % TODO --- Riformulare e approfondire meglio

Significant technological advancements have paved the way for the emergence of a field known as \emph{Collective Computing} 
    \cite{abowd2016beyond}, with \emph{Cyber-Physical Swarms (CPSW)} \cite{schranz2021swarm} as a noteworthy branch within it.
    The latter consist of myriad devices that interact with the environment and exchange information among themselves. 
    A crucial aspect of these systems is that a more complex collective behavior emerges from the interaction between 
    individual agents that leads to the resolution of various tasks.
    Among all aspects related to CPSW, our focus lies on properties like \emph{collective intelligence} \cite{tumer2004survey} 
    and \emph{self-organization} \cite{schmeck2011organic}. This stems from the applications of these systems, leading us to 
    concentrate on their collective behavior to express autonomy, adaptability, and coordination of the devices 
    that are part of them.

This progress has been driven by research in various related fields such as: multi-agent systems \cite{dorri2018multi},
     coordination \cite{yang2022overview}, distributed artificial intelligence \cite{bond2014readings}, and many others. 
     Additionally, it has a profound impact on a wide range of applied domains, including: smart cities \cite{zedadra2019swarm}, 
     swarm robotics \cite{brambilla2013swarm}, large-scale IoT systems \cite{uslu2023role}, and more.

A crucial aspect to consider in CPSW is how individual devices are programmed and achieve coordination to perform assigned tasks. 
Novel approaches -- like \emph{aggregate computing} \cite{viroli2018field} -- have focused on manually developing
controllers from a global perspective. However, this approach has some drawbacks: it is highly challenging to write satisfactory 
and efficient programs for complex tasks, they may be error-prone and lack of generality.

On the other hand, there exists approaches that leverage various artificial intelligence (AI) techniques, 
such as \emph{Multi-Agent Reinforcement Learning} (MARL) \cite{busoniu2008comprehensive},
to enable devices to learn directly from experience and/or data. These approaches also present several challenges, including: non-stationarity 
\cite{hernandez2017survey}, communication and scalability.

%
\paragraph{Thesis objectives}

Starting from what was seen in the previous paragraph, the goal of this thesis is to lay the foundation for a hybrid approach 
    that can succeed in exploiting the potential of both macro-programming, in particular \emph{aggregate computing} is taken as a 
    reference, and AI approach.
    In order to achieve this goal, it is necessary to develop a toolchain that allows these systems to be developed in an agile,
    fast and reusable way. 
    Scarlib, whose development has already started in \cite{scarlib}, is the tool that for us forms the basis of this toolchain.
    Its main purpose is to integrate the \emph{ScaFi} \cite{casadei2022scafi} (an implementation of aggregate computing) 
    and \emph{Alchemist} \cite{pianini2013chemical} (a bio-chemical based simulator) tools with \emph{Reinforcement Learning} 
    to help develop experiments in simulated environments with offline learning.

%
\paragraph{Thesis Structure} 


%----------------------------------------------------------------------------------------
\chapter{Background}
\label{chap:background}
%----------------------------------------------------------------------------------------

\section{Cyber-Physical Swarms}
%
\section{Aggregate Computing}
%
\section{Reinforcement Learning}
%
\section{Multi-Agent Reinforcement Learning}

%----------------------------------------------------------------------------------------
\chapter{Requirements} 
\label{chap:requirements}
%----------------------------------------------------------------------------------------


%----------------------------------------------------------------------------------------
\chapter{Design} 
\label{chap:design}
%----------------------------------------------------------------------------------------


%----------------------------------------------------------------------------------------
\chapter{Implementation} 
\label{chap:implementation}
%----------------------------------------------------------------------------------------


%----------------------------------------------------------------------------------------
\chapter{Validation} % possible chapter for Projects
\label{chap:validation}
%----------------------------------------------------------------------------------------


%----------------------------------------------------------------------------------------
\chapter{\conclusionsname}
\label{chap:conclusions}
%----------------------------------------------------------------------------------------


%----------------------------------------------------------------------------------------
% BIBLIOGRAPHY
%----------------------------------------------------------------------------------------

%\nocite{*} % uncomment this to show all the reference in the .bib file
\bibliographystyle{plain}
\bibliography{bibliography}


\end{document}